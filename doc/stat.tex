\documentclass{article}

\begin{document}
\section{Introduction and definitions}

We assume for simplicity only two moop values, $m=\{0,1\}$. 
These can be set to any combination of {\sc GREEN}, {\sc YELLOW} or {\sc RED}.
We define as $p_m(m)$ the probability of having a MOOP value $m$. 
This can be computed from the MOOP map assuming a population that is proportional to the geografic area. 

We assume only two categories, $y=\{0,1\}$, of population.
These could be for instance being at the first burn ($y=0$) or having being there before at least once ($y=1$).
We define $p_y(y)$ the probability of a given individual to be in the population category $y$.
This is derived from the standard census data analysis.

The population of individuals is distributed on the territory such that each individual is in principle
associated to a category $y$ and a MOOP value $m$ based on its true position $x_{\rm true}$.

We are interested in studying the possible correlation between the variables $m$ and $y$ across the population,
that is if the probability of an individual to be associated to a MOOP value $m$ depends on the category $y$, 
or in other words whether $p(m | y) = p(m)$ or not.

We define for simplicity a ``MOOP asymmetry'' as:
\begin{equation}
\label{eq:moopasym}
a \equiv 2\cdot \frac{p(m=1 | y=0) - p(m=1 | y=1)}{p(m=1 | y=0) + p(m=1 | y=1)}
\end{equation}
 which quantifies how more likely is an individual of category $m=1$ to also be in category $y=0$ rather than $y=1$.
In the example above, this would correspond to the difference of probability to produce MOOP ($m=1$) for a new-burner ($y=0$)
and an experienced one ($y=1$), divided by its average. An asymmetry value of $0$ would indicate that 
virgin and experienced burners have the same probability to produce MOOP, for instance.

\section{Toy generation}
The aim is to generate a random population of individuals that are distributed randomly on the places of the MOOP map
that are allowable for camping.
The population shall respect the constraints given by $p_y$ and $p_m$ and is generated such that it will have some
amount of correlation between the $m$ and $y$ variables in the population given by the aymmetry $a$, as defined in equation \ref{eq:moopasym}.

We generate toys to study the performance of a given algorithm and how much external factors (e.g. biases, measurement uncertainties)
influence the significance of the result. 
For each toy ``experiment'' we generate a full population of a given number of individuals.
For each individual we:
\begin{itemize}
\item generate a true position that is fiducial to the available area for camping
\item smear the true position through a given algorithm to emulate accuracy in the measurements
\item assign a category $y$, based on the conditional probability $p(y | m_{\rm true})$, where $m_{\rm true}$ is the true value of $m$ at the generated true position (before smearing)
\end{itemize}

The true position is trivially generated in a way that the probability of finding an individual in an area $dx\cdot dy$ is constant within the fiducial area.
The fiducial area is defined as the area available for camping.

The smearing can be done in different ways. Three possibilities that are well motivated include:
\begin{itemize}
\item[No smearing] Useful mostly for studying how other variables affect sensitivity.
\item[Gaus] Gaussian smearing around the true value with a given standard deviation.
\item[Glue] Each psition is collapsed to the closest intersection. Some uncertainty can be added to emulate mistakes in filling this information.
\end{itemize}

The value of $p(y | m)$ needed to assign a category for any given generated individual can be calculated from the asymmetry $a$, and the overall
probabilities $p_m(m)$ and $p_y(y)$. 
The calculation can be carried out using Bayes theorem to express:
\begin{equation}
\label{eq:bayes}
p(y | m) = p(m | y)\cdot p_y(y) / p_m(m)
\end{equation}
the values of $p(m | y)$ can be extracted using the fact that $\sum_y p(m | y)\cdot p(y) = \sum_y p(y | m)\cdot p(m) = p(m)$, which simply
uses Bayes theorem again and unitarity, i.e. $\sum_y p(y | m) = 1$.
For the case of $m=1$ (used to define the asymmetry), and calling $p(m=m_0 | y=y_0)\equiv p_{m_0y_0}$ for brevity, we have the 
following relationships:
\begin{eqnarray}
p_{10}p_y(0) + p_{11}p_y(1) & = & p_m(1) \\
\frac{p_{11}}{p_{10}} & = & \frac{2-a}{a+2} \equiv A \\
\end{eqnarray}
where the second one comes straight from the asymmetry definition of equation \ref{eq:moopasym}.
We also defined the useful symbol $A$, which relates the ratio $p(m=1|y=1) / p(m=1|y=0)$, as alternative interpretation of the asymmetry
\footnote{While in intuitive terms this seems a more natural definition, the definition of $a$ is slightly preferred, since non-small deviation influence $A$ in the same way on both sides, while large variations of $a$ leads to values in the asymmetric ranges $[0,1]$ and $[1,\inf]$. The two can easily be converted to each other at any time though.}.
These, plus unitarity, leads to the determination of:
\begin{eqnarray}
p(m=1 | y=0) &=& \frac{p_m(1)}{p_y(0) + A\cdot p_y(1)} \\
p(m=1 | y=1) &=& \frac{p_m(1)}{p_y(0)/A + p_y(1)} \\
p(m=0 | y=0) &=& 1-p(m=1 | y=0)) \\
p(m=0 | y=1) &=& 1-p(m=1 | y=1)) \\
\end{eqnarray}

We can therefore finally write the needed probability expressions combining the above equations with eq.~\ref{eq:bayes}:
\begin{eqnarray}
p(y=1 | m=1) = \frac{p_y(1)}{p_y(0)/A + p_y(1)} \\
p(y=0 | m=1) = \frac{p_y(0)}{p_y(0) + A\cdot p_y(1)} \\
p(y=0 | m=0) = \frac{p_y(0)}{p_m(0)}\left(1- \frac{p_m(1)}{p_y(0)+A\cdot p_y(1)}\right) \\
p(y=1 | m=0) = \frac{p_y(1)}{p_m(0)}\left(1- \frac{p_m(1)}{p_y(0)/A+p_y(1)}\right) \\
\end{eqnarray}
\noindent
with $A = \frac{2-a}{a+2}$.

In practical terms we, only need two of them (e.g. $p(y=0 | m=1)$ and $p(y=0 | m=0)$ to generate a random number and decide for each individual of class $m$ what category $y$ to assign.

\end{document}
